
% Capitulos sao secoes
% Artigos e paragrafos sao enumerates

\documentclass{article}
\usepackage{enumitem}
\usepackage[utf8]{inputenc}
\usepackage{titlesec}

\titleformat{\section}{\normalfont\scshape}{Capítulo \Roman{section}}{1em}{}

\title{Regimento PPGC}

\begin{document}

\maketitle

\section{Das finalidades}


\begin{enumerate}%[label=Art. \thesection]
\item O Programa de Pós-Graduação em Computação da Universidade Federal de Pelotas, neste documento referenciado por PPGC ou simplesmente por Programa, em nível de Mestrado e Doutorado, tem por finalidade a formação de recursos humanos para o ensino e pesquisa capazes de realizar projetos de investigação científica, incluindo aspectos de planejamento, delineamento, execução, análise e publicação, contribuindo com o avanço do conhecimento científico e tecnológico em Computação.
\end{enumerate}

\section{Da Administração do Programa}

\begin{enumerate}

	\item O PPGC é administrado pelo Colegiado do Programa, presidido pelo Coordenador do Programa. 
	\begin{enumerate}
		\item Na ausência do Coordenador, preside o Colegiado o Coordenador Adjunto ou, na ausência deste, pelo membro mais antigo na Instituição pertencente ao Colegiado.
	\end{enumerate}

	\item O Coordenador e o Coordenador Adjunto são eleitos entre os membros do corpo docente do Programa através de votação pelo Colegiado. 
	\begin{enumerate}
		\item O Coordenador será o membro com o maior número de votos.
		\item O Coordenador Adjunto será o membro com o segundo maior número de votos.
	\end{enumerate}

	\item O Colegiado é composto pelo Coordenador, por um representante discente e por docentes do programa em número igual a quarta parte do total de docentes.
	\begin{enumerate}
		\item Os membros docentes do Colegiado são eleitos pelo corpo docente do Programa através de votação.
		\item Cada Linha de Pesquisa terá direito a uma posição no Colegiado, na forma do seu membro com maior número de votos.
		\item As demais posições serão preenchidas na ordem especificada na votação, do docente com maior número de votos para o com menor número.
		\item A quarta parte, resultando em número fracionário, será arredondado para o número inteiro mais próximo.
		\item Votos ao Coordenador devem ser excluídos do processo.
	\end{enumerate}
\end{enumerate}

\section{Das Atribuições do Colegiado}
\begin{enumerate}

	\item O Colegiado é o órgão superior do Programa, com funções normativas, deliberativas e de supervisão.
	\item O Colegiado do Programa deliberará por maioria simples de votos dos membros presentes.
	\begin{enumerate}
		\item O Coordenador do Colegiado não possui direito a voto.
		\item Todos os demais membros do Colegiado possuem direito a voto, em igual peso.
		\item O Coordenador dará o Voto de Qualidade em caso de empate na votação. % Isso eh diferente de dizer que o Coordenador tem direito a voto apenas em caso de empate?
	\end{enumerate}


	\item Compete ao Colegiado do Programa:
	\begin{enumerate}
		\item Executar as diretrizes estabelecidas pela Pró-Reitoria de Pesquisa e Pós-Graduação e pelo Conselho Coordenador de Ensino, Pesquisa e Extensão desta Instituição;
		\item Exercer a coordenação interdisciplinar, visando a conciliar os interesses de ordem didática dos departamentos com o do Programa de Pós-Graduação;	% Não temos mais departamentos!
		\item Elaborar e manter atualizadas as informações didáticas do Programa;
		\item Fixar a sequência recomendável de estudos e os pré-requisitos necessários;
		\item Emitir parecer sobre assuntos de interesse do Programa de Pós-Graduação;
		\item Analisar e emitir parecer sobre os pedidos de transferência, aproveitamento de estudos e adaptações, de acordo com as normas fixadas pelo Conselho Coordenador de Ensino, Pesquisa e Extensão e a regulamentação estabelecida pelo Conselho de Pós-Graduação;
		\item Julgar, em grau de recurso, decisões proferidas pelo Coordenador do Programa;
		\item Elaborar o Regimento do Programa de Pós-Graduação contendo as normas relativas ao funcionamento do mesmo, para aprovação pela Câmara de Pós-Graduação ``Stricto Sensu'' e pelos demais órgãos competentes;
		\item Verificar o cumprimento do Conteúdo Programático e da Carga Horária das disciplinas do curso; % Cumprimento por quem? Docentes ou alunos?
		\item Estabelecer mecanismos de orientação acadêmica aos mestrandos do curso;
		\item Acolher, avaliar, solicitar alterações e aprovar o plano de estudo de cada Mestrando antes do final do primeiro período letivo; % Tem que adaptar para o doutorado!
		\item Promover o acompanhamento dos mestrandos por meio de registros individuais;
		\item Homologar as dissertações após as correções sugeridas pela banca examinadora;
		\item Homologar a nominata para Banca Examinadora de cada pedido de Defesa de Dissertação recebido;
		\item Instalar, anualmente, uma Comissão de Seleção de Ingresso para encaminhamento do Processo Seletivo de candidatos ao ingresso no Programa.
		\item Indicar, à ocasião do Processo Seletivo de novos ingressantes, os Orientadores para cada candidato selecionado; % Não está incluso no 'Estabelecer mecanismos de orientação'?
		\item Propor aos órgãos competentes da Universidade a interrupção, suspensão ou cessação das atividades do Programa;
		\item Avaliar anualmente o desempenho global do PPGC e ao término de cada período de três anos, ou à ocasião da avaliação trienal da CAPES, realizar a avaliação do Corpo Docente promovendo o descredenciamento de membros junto a Programa; % Agora é quadrienal. Mas tem que estar fixo no regimento?
		\item Instalar anualmente e por ocasião da avaliação trienal da CAPES um Comitê de Avaliação do Programa representado por membros de todas as linhas existentes no Programa e apreciar o Relatório gerado por este Comitê; % Quadrienal...
		\item Manifestar-se sobre as Regras de Avaliação do Programa e as Regras de Avaliação Docente propostas pelo Comitê de Avaliação;
		\item Analisar e se pronunciar sobre o Relatório de Avaliação do Programa e propor ações cabíveis para melhora de sua qualidade;
		\item Receber, avaliar e apresentar julgamento sobre pedidos de credenciamento de docentes junto ao Programa;
		\item Se pronunciar sobre prioridades de aplicação de recursos específicos do Programa;
		\item Reunir-se para escolha de novo Coordenador de Programa e Coordenador Adjunto quando terminado o mandato de dois anos ou no descredenciamento do primeiro; % Primeiro?
		\item Resolver, nos limites de sua competência, os casos omissos deste Regimento.
		\item Recursos às decisões do Colegiado de Programa devem ser dirigidos à Câmara de Pós-Graduação ``Stricto Sensu'' da Pró-reitoria de Pesquisa e Pós-Graduação desta Universidade.

	\end{enumerate}
\end{enumerate}

\section{Do Coordenador do Programa}
\begin{enumerate}
	% Acho que o item abaixo tem que ir para a Da Administração do Programa e esta seção se tornar "Atribuições da Coordenação"
	\item O Programa terá um Coordenador que deverá ser membro do seu Colegiado e Docente da Universidade Federal de Pelotas, ser eleito pelo voto universal dos membros do Colegiado e de acordo com norma específica do Regimento Geral dos Cursos de Pós-Graduação ``Stricto Sensu'' desta Instituição.
	\begin{enumerate}
		\item O Coordenador terá mandato de dois anos e será permitida apenas uma recondução sucessiva ao cargo.
	\end{enumerate}

	\item Ao Coordenador de Programa, compete:
	\begin{enumerate}
		\item Coordenar e supervisionar o funcionamento do Programa;
		\item Convocar e presidir as reuniões do Colegiado do Programa, com direito ao voto de qualidade;
		\item Representar o Colegiado e as decisões tomadas neste fórum;
		\item Enviar, semestralmente, à Pró-Reitoria de Pesquisa e Pós-Graduação, de acordo com o calendário vigente, ouvidos os Departamentos e professores envolvidos, a relação de disciplinas a serem ofertadas com os respectivos professores responsáveis;
		\item Enviar à Pró-Reitoria, em tempo oportuno, as necessidades de bolsas, bem como sua distribuição entre os discentes;
		\item Elaborar os relatórios anuais destinados às instituições fornecedoras de bolsas, enviando-os à Pró-Reitoria de Pesquisa e Pós-Graduação;
		\item Comunicar ao órgão competente qualquer irregularidade no funcionamento do Programa e solicitar as correções necessárias;
		\item Designar Relator ou Comissão para estudo de matéria submetida ao Colegiado;
		\item Articular o Colegiado com os Departamentos e outros órgãos envolvidos;
		\item Decidir sobre matéria de urgência ``ad referendum'' do Colegiado;
		\item Exercer outras atribuições inerentes ao cargo;
		\item Supervisionar e zelar pela aplicação das verbas específicas do Programa.

	\end{enumerate}

	\item Ao Coordenador Adjunto de Programa, compete Substituir o Coordenador em suas ausências ou impedimentos, auxiliá-lo na execução das deliberações do Colegiado e executar as tarefas que lhe forem especificamente designadas pelo Colegiado.
\end{enumerate}

\section{Do Corpo Docente}

\begin{enumerate}
	\item O Corpo Docente do PPGC é constituído pelos professores do quadro Permanente, portadores de título de doutor, que são responsáveis por ministrar disciplinas regulares no Programa e estão habilitados a orientar e co-orientar dissertações.

	\begin{enumerate}
		\item O Corpo Docente do Programa deve ser constituído, majoritariamente, por docentes da Universidade Federal de Pelotas.
		\item Poderão integrar o Corpo Docente do Programa, inclusive, como Professor Responsável de Disciplina, professores portadores de título de doutor, de outras Instituições de Ensino Superior, nacionais ou estrangeiras, de centros de pesquisa, bem como outros profissionais portadores de título de doutor, do país ou do exterior.
		\item Para solicitar credenciamento junto ao Corpo Docente do Programa o interessado deve possuir produção científica relevante na área. A solicitação deve ser encaminhada a qualquer tempo ao Colegiado, acompanhada de curriculum vitae.
	\end{enumerate}

	\item Para efeito de credenciamento junto ao Programa, os docentes serão designados como:
	\begin{enumerate}
		\item Permanentes – aqueles que atuam com preponderância no curso, de forma mais direta, intensa e contínua, constituindo o núcleo estável de docentes que desenvolvem as principais atividades de ensino, orientando dissertações e pesquisas do Programa.
		\item Visitantes – identificados por estarem vinculados à própria Universidade Federal de Pelotas ou a outra instituição de Ensino Superior, no Brasil ou no Exterior, que permanecerem, durante um período contínuo e determinado, à disposição do Programa, contribuindo para o desenvolvimento das atividades acadêmico-científicas deste.
		\item Colaboradores – aqueles que contribuem para o curso de forma complementar ou eventual, auxiliando no desenvolvimento de disciplinas, co-orientando dissertações, colaborando em projetos de pesquisa, sem que, todavia, tenham carga intensa e permanente no programa.
	\end{enumerate}

	\item Não pertencem ao Colegiado do Programa os professores Visitantes e Colaboradores. % Moot

	\item São atribuições dos docentes:
	\begin{enumerate}
		\item	Ministrar aulas teóricas e práticas de disciplinas do Programa, de acordo com o programa vigente de cada Disciplina;
		\item	Manter o Registro Acadêmico da Disciplina, bem como o Registro de Desempenho individual de cada Aluno nela inscrito;
		\item	Atualizar o programa da disciplina a cada edição desta;
		\item	Atuar como Professor Orientador ou Co-orientador em Dissertações de Mestrado e Teses de Doutorado de alunos do Programa;
		\item	Promover e participar de seminários, simpósios e estudos dirigidos;
		\item	Participar de Comissões Examinadoras;
		\item	Estar ativamente envolvido em pesquisas na área de Computação;
		\item	Desempenhar demais atividades, dentro dos dispositivos regimentais, que possam beneficiar o Programa. 
		\item	Responder a Comissão de Avaliação do Programa quando solicitado;
		\item	Desenvolver pesquisa que resulte em produção científica divulgada em periódicos indexados;
		\item	Divulgar resultados de pesquisas em eventos qualificados;
		\item	Promover integração com a região prestigiando eventos científicos regionais;
		\item	Promover a pesquisa em Computação nos cursos de graduação da área;
		\item	Acatar as decisões do Colegiado e executar as tarefas que neste fórum lhe foram atribuídas no prazo conveniado;
		\item	Integrar o Colegiado do Curso.
	\end{enumerate}

	\item É assegurada ao Docente autonomia didática, nos termos da legislação vigente, do regimento da Universidade Federal de Pelotas e deste Regimento.

\end{enumerate}

\section{Da Orientação}

\begin{enumerate}
	\item Cada Aluno ingressante no Programa contará com um Orientador e deverá se reportar a um Comitê de Acompanhamento.
\end{enumerate}
\end{document}